\documentclass[a4paper, twocolumn]{article}
\usepackage{amsmath}
\usepackage{amssymb}
\usepackage{kotex}
\usepackage{graphicx}
\usepackage{hyperref}
%opening
\title{단진자의 가속도는 어디에서 최대가 될까?}
\author{류명선\\\href{mailto:mingshey@hafs.hs.kr}{mingshey@hafs.hs.kr}\\{용인한국외국어대학교부설고등학교}}
\date{2020-04-22}
\date{}
\begin{document}
\textheight=260mm
\textwidth=180mm
\voffset=-25mm
\topmargin=0mm
\oddsidemargin=-10mm

\maketitle

\parskip=1ex
\parindent=0em
\baselineskip=14pt
단진자의 가속도는 구심 가속도와 접선 가속도 성분이 있다. 접선 방향 가속도는 평형점에서 0이 되며, 이 때 구심 가속도는 최대가 된다. 반면 최고 높이에서 속력이 0이므로 구심 가속도는 0이고 접선 방향 가속도는 기울기가 $ 90^\circ $일 때 최대가 된다.  그렇다면 두 가속도 성분을 합성한 값은 어느 지점에서 최대가 될까? 14기 김한재, 허정민 학생이 제기한 이 질문에 답해 보기로 한다.

먼저, 길이 $l$, 추의 질량 $m$인 단진자의 각진폭이 $\Theta$, 즉 $$ -\Theta\le\vartheta\le\Theta $$라 하고, 단진자가 각 $\vartheta$만큼 기운 순간 접선 가속도의 크기를 $a_t$, 구심 가속도의 크기를 $a_c$라 하자.\par
\hfill \includegraphics[width=0.8\linewidth]{pendulum.eps} \hfill\,

그러면, 역학적 에너지 보존으로부터 $$ mgl\left(1-\cos\Theta\right)=\dfrac{1}{2}m v^2 + mgl\left(1-\cos\vartheta\right) $$와 구심 가속도의 식 $$ a_c = \dfrac{v^2}{l} $$을 이용하면
\begin{align}
a_t &= g\sin\vartheta\label{eqn:at}\\
a_c &= \dfrac{v^2}{l}=2g\left(\cos\vartheta-\cos\Theta\right)\label{eqn:ac}
\end{align}
이다. 

$u\equiv\cos\vartheta$, $U\equiv\cos\Theta$로 두면, $\sin\vartheta=\sqrt{1-u^2}$이고, 
\begin{equation} U<u\le{}1.\label{eqn:U-range} \end{equation} 
피타고라스의 정리를 이용하여 합성한 가속도의 절댓값은,
\begin{align*}
a&=\sqrt{a_t^2 + a_c^2}=g\sqrt{1-u^2 + 4 u^2-8 U u+4 U^2}\\
a(u) &=g\sqrt{3 u^2-8 U u + 4U^2 +1}
\end{align*}
이다.

최댓값을 구하기 위해 먼저 극값을 구해 보면, $$  \dfrac{du}{d\vartheta} =-\sin\vartheta $$이고,
\begin{align}
\dfrac{da(\vartheta)}{d\vartheta}&=\dfrac{da}{du}\dfrac{du}{d\vartheta}\\
&=-\dfrac{6 u- 8U}{2\sqrt{3 u^2-8 U u + 4U^2 +1}}\sin\vartheta\nonumber \\
&=\dfrac{(4\cos\Theta-3\cos\vartheta)\sin\vartheta}{\sqrt{3 u^2-8 U u + 4U^2 +1}}\nonumber \\
&=0\nonumber
\end{align}
으로부터,
\begin{align}
\sin\vartheta&=0, \textrm{ 또는}\label{eqn:sin-theta-0}\\
\cos\vartheta &= \dfrac{4}{3}\cos\Theta=\dfrac{4}{3}U.\label{eqn:cos-theta-4/3U}
\end{align}
일 때 극값을 가진다. (\ref{eqn:U-range})에서, (\ref{eqn:cos-theta-4/3U})의 극값을 가지는 것은 $0\le{}U\le\dfrac{3}{4}$일 때임을 알 수 있다.\par

각각의 경우 가속도 값은,\par

i)  (\ref{eqn:sin-theta-0}) $\sin\vartheta = 0$인 경우, 즉 $\vartheta=0$인 경우,
\begin{align}
a_t&=0\nonumber\\
a_c&=2g(1-\cos\Theta),\nonumber\\
\therefore a_0 &= a_c=2g(1-\cos\Theta)\nonumber\\
             &=2g(1-U).
\end{align}
ii) (\ref{eqn:cos-theta-4/3U}) $u=\dfrac{4}{3}U$, 즉 $\cos\vartheta = \dfrac{4}{3}\cos\Theta$인 경우, (이 때 $\left|\cos\Theta\right|\le\dfrac{3}{4}$이어야 한다.)
\begin{align}
a_t &= g\sin\vartheta=g\sqrt{1-\cos^2\vartheta}=g\sqrt{1-\dfrac{16}{9}U^2}.\nonumber\\
a_c &= 2g\left(\dfrac{4}{3}U-U\right)= \dfrac{2}{3}gU.\nonumber\\
\therefore a_1 &= \sqrt{\dfrac{4}{9}g^2U^2 + g^2\left(1-\dfrac{16}{9}U^2\right)}\nonumber\\
             &= g\sqrt{1-\dfrac{4}{3}U^2}.\label{eqn:a1(u)}
\end{align}
$a_0$과 $a_1$ 중 어느 값이 더 큰지 비교해 보면,
$$ f_0(U)=\dfrac{a_0}{g}=2(1-U) $$와 $$ f_1(U)=\dfrac{a_1}{g}=\sqrt{1-\dfrac{4}{3}U^2} $$은 $U=\dfrac{3}{4}$일 때 한 점에서 만남을 간단히 증명할 수 있고(연습문제), $ U=0 $ 일 때 $f_0(0) > f_1(0)$이므로 항상 $f_0(U)\ge{}f_1(U)$임을 알 수 있다.\par
iii) 또한 구간 한계($\vartheta=\Theta$)에서의 가속도 값을 검토해 보면 (\ref{eqn:at}), (\ref{eqn:ac}) 에서
\[\left\{
\begin{array}{rl}
a_t &= g\sin\Theta=g\sqrt{1-U^2},\\
a_c &= 0
\end{array}
\right.\]
로부터,\par
\hfill $ a_2(U)=a_t=g\sqrt{1-U^2} $\hfill\,\par
이고,\par
\hfill $ f_2(U)=a_2/g=\sqrt{1-U^2} $\hfill\,\par
이라 하면, $ f_0(U) $와 $f_2(U)$는 $U=3/5$에서 교차하고, $U>3/5$일 때는 $f_2(U)>f_0(U)$이다.\par

따라서 가속도의 최댓값은 각진폭  $\Theta$에 대하여 \par
$\cos\Theta<3/5$일 때는 $\vartheta=0$일 때 $$ a_0=2g(1-\cos\Theta), $$\par $\cos\Theta=3/5$일 때는 $\vartheta=\Theta$ 및 $\vartheta=0$일 때 같은 최댓값 $$ a_0=a_2=\dfrac{4}{5}g, $$\par
$\cos\Theta>3/5$일 때는 $\vartheta=\Theta$일 때 $$ a_2=g\sin\Theta $$이다.\par
\hfill \includegraphics[]{Simple_Pendulum.eps} \hfill\,\par
(연습 문제) 식 (\ref{eqn:a1(u)})의 극값 $a_1(U)$가 극솟값임을 보일 수 있겠는가? 
\end{document}
